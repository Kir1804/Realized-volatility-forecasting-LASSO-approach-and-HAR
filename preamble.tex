% Общее:
\usetheme{CambridgeUS} % колонтитулы, углы, и другая красота
% Другие темы можно найти на https://latex-beamer.com/tutorials/beamer-themes/
\usecolortheme{seahorse} % но по факту мы почти все цвета сами задаем
% https://deic.uab.cat/~iblanes/beamer_gallery/index_by_color.html
\usefonttheme{professionalfonts}

\usepackage {mathtools} % красивая математика
\usepackage{amsmath,amsfonts,amssymb,amsthm,mathtools} % еще математика

\usepackage[
backend=biber,
style=numeric,
sorting=nty
]{biblatex} % для библиографии
\addbibresource{references.bib}
\usepackage{hyperref} % для ссылок



% Создаем цвета
\definecolor{darkredNES}{RGB}{115,15,15} % тёмно-красный
\definecolor{darkblueNES}{RGB}{20,20,100} % тёмно-синий
\definecolor{redNES}{RGB}{170,35,35} % красный посветлее
\definecolor{darkgreenNES}{RGB}{7,110,7} % тёмно-зеленый
\definecolor{alertredNES}{RGB}{125,25,25} % красный средний
% Однако в основном проще пользоваться стандартными цветами, подробнее тут:
% https://www.overleaf.com/learn/latex/Using_colours_in_LaTeX

\setbeamercolor*{palette primary}{bg=darkredNES} %правые рамки
\setbeamercolor*{palette secondary}{bg=darkredNES, fg = white} %центральные
\setbeamercolor*{palette tertiary}{bg=darkredNES, fg = white} %левые

\setbeamercolor*{titlelike}{fg=darkredNES} % названия слайдов
\setbeamercolor*{title}{bg=darkredNES, fg = white} % титул и отделы
\setbeamercolor*{item}{fg=redNES} % для списков (например, в оглавлении)
\setbeamercolor*{caption name}{fg=darkblueNES} % названия картинок
\setbeamercolor{alerted text}{fg=alertredNES} % выделенка

\setbeamertemplate{blocks}[rounded, shadow=true]
\setbeamercolor{block title}{bg=darkblueNES, fg=white}
\setbeamercolor{block title alerted}{bg=alertredNES, fg=white}
\setbeamercolor{block title example}{bg=darkgreenNES, fg=white}

\setbeamercolor{bibliography entry author}{fg=black}
\setbeamercolor{bibliography entry title}{fg=black}
\setbeamercolor{bibliography entry note}{fg=darkblueNES}

\setbeamercolor{page number in head/foot}{fg=white}




% Для русского нам пригодится
\usepackage[english,russian]{babel} % локализация и переносы
\usepackage{fontspec}
\usepackage[T2A]{fontenc} % кодировка
\usepackage[utf8]{inputenc} % кодировка исходного текста
\usepackage{cmap} % поиск в PDF
\usepackage{mathtext} % русские буквы в формулах



% Шрифты
\setsansfont{Times New Roman} % Настройка Шрифта
% \setsansfont{Noto Sans} этот можете использовать, если вам не нравятся засечки на буквах
% \setmainfont{Arial} % Дополнительно
% \setmonofont{Arial} % Нужно Разобраться за что они отвечают
% Другие шрифты можно поскачивать на https://www.ctan.org/tex-archive/fonts



% Работа с картинками
\usepackage{graphicx} % Для вставки рисунков
\setlength\fboxsep{3pt} % Отступ рамки \fbox{} от рисунка
\setlength\fboxrule{1pt} % Толщина линий рамки \fbox{}
\usepackage{wrapfig} % Обтекание рисунков текстом
\DeclareGraphicsExtensions{.pdf,.png,.jpg,.HEIC} % работа с форматами
% \graphicspath{{images/}} % папки с картинками, в оверлифе это не нужно
% Ещё о картинках https://www.overleaf.com/learn/latex/Inserting_Images



% Работа с таблицами
\usepackage{array,tabularx,tabulary,booktabs} % Дополнительное для таблиц
\usepackage{longtable} % Длинные таблицы
\usepackage{multirow} % Слияние строк в таблице



% Свои команды, если лень прописывать \mathbb итп.
\def\E{\exists} % существует
\def\A{\forall} % для всех

\def\La{\mathcal{L}} % калиграфическое L (Лагранжиан, преобразования Лапласа)
\def\la{\lambda} % лямбда
\def\a{\alpha} % альфа
\def\b{\beta} % бэта
\def\e{\varepsilon} % эпсилон (покрасивее)
\let\phi\varphi % фи (покрасивее)

\def\N{\mathbb{N}} % натуральные
\def\Z{\ensuremath{\mathbb{Z}}} % целые
\def\R{\mathbb{R}} % рациональные
\def\C{\mathbb{C}} % комплексные

\def\~{\sim} % подобно



%Возможно, у вас возникнут проблемы с буквой @, попробуйте одно из следующих:
%\makealetter
%\makeatother