\documentclass[11pt]{beamer} % Определяем тип документа как презентацию
% в квадратных скобках размер шрифта: 8pt, 9, 10, 11 (def), 12, 14, 17, 20.

% Общее:
\usetheme{CambridgeUS} % колонтитулы, углы, и другая красота
% Другие темы можно найти на https://latex-beamer.com/tutorials/beamer-themes/
\usecolortheme{seahorse} % но по факту мы почти все цвета сами задаем
% https://deic.uab.cat/~iblanes/beamer_gallery/index_by_color.html
\usefonttheme{professionalfonts}

\usepackage {mathtools} % красивая математика
\usepackage{amsmath,amsfonts,amssymb,amsthm,mathtools} % еще математика

\usepackage[
backend=biber,
style=numeric,
sorting=nty
]{biblatex} % для библиографии
\addbibresource{references.bib}
\usepackage{hyperref} % для ссылок



% Создаем цвета
\definecolor{darkredNES}{RGB}{115,15,15} % тёмно-красный
\definecolor{darkblueNES}{RGB}{20,20,100} % тёмно-синий
\definecolor{redNES}{RGB}{170,35,35} % красный посветлее
\definecolor{darkgreenNES}{RGB}{7,110,7} % тёмно-зеленый
\definecolor{alertredNES}{RGB}{125,25,25} % красный средний
% Однако в основном проще пользоваться стандартными цветами, подробнее тут:
% https://www.overleaf.com/learn/latex/Using_colours_in_LaTeX

\setbeamercolor*{palette primary}{bg=darkredNES} %правые рамки
\setbeamercolor*{palette secondary}{bg=darkredNES, fg = white} %центральные
\setbeamercolor*{palette tertiary}{bg=darkredNES, fg = white} %левые

\setbeamercolor*{titlelike}{fg=darkredNES} % названия слайдов
\setbeamercolor*{title}{bg=darkredNES, fg = white} % титул и отделы
\setbeamercolor*{item}{fg=redNES} % для списков (например, в оглавлении)
\setbeamercolor*{caption name}{fg=darkblueNES} % названия картинок
\setbeamercolor{alerted text}{fg=alertredNES} % выделенка

\setbeamertemplate{blocks}[rounded, shadow=true]
\setbeamercolor{block title}{bg=darkblueNES, fg=white}
\setbeamercolor{block title alerted}{bg=alertredNES, fg=white}
\setbeamercolor{block title example}{bg=darkgreenNES, fg=white}

\setbeamercolor{bibliography entry author}{fg=black}
\setbeamercolor{bibliography entry title}{fg=black}
\setbeamercolor{bibliography entry note}{fg=darkblueNES}

\setbeamercolor{page number in head/foot}{fg=white}




% Для русского нам пригодится
\usepackage[english,russian]{babel} % локализация и переносы
\usepackage{fontspec}
\usepackage[T2A]{fontenc} % кодировка
\usepackage[utf8]{inputenc} % кодировка исходного текста
\usepackage{cmap} % поиск в PDF
\usepackage{mathtext} % русские буквы в формулах



% Шрифты
\setsansfont{Times New Roman} % Настройка Шрифта
% \setsansfont{Noto Sans} этот можете использовать, если вам не нравятся засечки на буквах
% \setmainfont{Arial} % Дополнительно
% \setmonofont{Arial} % Нужно Разобраться за что они отвечают
% Другие шрифты можно поскачивать на https://www.ctan.org/tex-archive/fonts



% Работа с картинками
\usepackage{graphicx} % Для вставки рисунков
\setlength\fboxsep{3pt} % Отступ рамки \fbox{} от рисунка
\setlength\fboxrule{1pt} % Толщина линий рамки \fbox{}
\usepackage{wrapfig} % Обтекание рисунков текстом
\DeclareGraphicsExtensions{.pdf,.png,.jpg,.HEIC} % работа с форматами
% \graphicspath{{images/}} % папки с картинками, в оверлифе это не нужно
% Ещё о картинках https://www.overleaf.com/learn/latex/Inserting_Images



% Работа с таблицами
\usepackage{array,tabularx,tabulary,booktabs} % Дополнительное для таблиц
\usepackage{longtable} % Длинные таблицы
\usepackage{multirow} % Слияние строк в таблице



% Свои команды, если лень прописывать \mathbb итп.
\def\E{\exists} % существует
\def\A{\forall} % для всех

\def\La{\mathcal{L}} % калиграфическое L (Лагранжиан, преобразования Лапласа)
\def\la{\lambda} % лямбда
\def\a{\alpha} % альфа
\def\b{\beta} % бэта
\def\e{\varepsilon} % эпсилон (покрасивее)
\let\phi\varphi % фи (покрасивее)

\def\N{\mathbb{N}} % натуральные
\def\Z{\ensuremath{\mathbb{Z}}} % целые
\def\R{\mathbb{R}} % рациональные
\def\C{\mathbb{C}} % комплексные

\def\~{\sim} % подобно



%Возможно, у вас возникнут проблемы с буквой @, попробуйте одно из следующих:
%\makealetter
%\makeatother 
% В преамбуле можно сделать небольшие изменения (цвет, шрифт итд.) Если есть желание изменить больше, возможно, проще выбрать другой шаблон.

% Начнем работу с презентацией

%Для начала подготовим титульный лист и оглавление
%Логотип РЭШ
\titlegraphic{\includegraphics[height=1.5cm]{}}

%Размеры шрифтов титульного листа; Цвета определены в преамбуле
\setbeamerfont{title}{size=\huge}
\setbeamerfont{subtitle}{size=\large}
\setbeamerfont{author}{size=\normalsize}
\setbeamerfont{date}{size=\normalsize}
\setbeamerfont{institute}{size=\normalsize}
% Больше о шрифтах вы найдете в том числе по следующей ссылке:
% https://tex.stackexchange.com/questions/183052/what-are-all-the-possible-first-arguments-to-setbeamerfont

%Тут квадратные скобки --- всё, что будет внизу странички
\title[ЦМФ]{RV и ее применение в инвестиционных стратегиях}
\subtitle{Проект}
\author[Кирилл Дорцев]{Кирилл Дорцев}
\institute[]{Центр Математических Финансов}
\date[\textcolor{white}{\today}]{\today}



% Следующее пригодится, если нужно показывать начало секции с оглавлением
%\AtBeginSection[]
%{
%  \begin{frame}
%    \frametitle{Contents}
%    \tableofcontents[currentsection]
%  \end{frame}
%}

\AtBeginSection[]{
  \begin{frame}
  \vfill
  \centering
  \begin{beamercolorbox}[sep=8pt,center,shadow=true,rounded=true]{title}
    \usebeamerfont{title}\insertsectionhead\par%
  \end{beamercolorbox}
  \vfill
  \end{frame}
}



\begin{document} % начнем саму презентацию

\frame{\titlepage} % сделаем титульный слайд и оглавление:

\begin{frame}
    \frametitle{Оглавление}
    \tableofcontents
\end{frame}



% Далее начнем работу со слайдами:

\section{Введение}
    \begin{frame}{Введение}
        \item 
    \begin{itemize}
        \item Проект основан на исследовании 2021 года [Ding et al, 2021], где авторы использовали различные модели для предсказания реализованной волатильности \\
        \item Изначальное исследование рассматривало 5-ти минутные показатели индексов нескольких стран
        \end{itemize}
    \end{frame}

\section{Содержание}
    \begin{frame}{Введение в RV}
        \centering
	% {\large На этом слайде все пункты появляются одновременно}

        \begin{itemize}
            \item В качестве оценки
волатильности и качества прогнозной силы в литературе и на практике широко используется «реализованная волатильность», которая получила широкое применение с появлением высокочастотных внутридневных данных котировок цен.

            \item RV_t = \sum\limits_{i=1}^n(r_t,_i)^2

            

            % \item r_t,_j = log(p_t,_i) - log(p_t_i_+_1 ) 

        \end{itemize}

        % {\large На самом деле можно делать и перечисление вместо списка}
        % \begin{enumerate}
        %     \item Первый

        %     \item Второй

        %     \item Третий
        % \end{enumerate}
    \end{frame}
    \begin{frame}{Содержание 2}
        \frametitle{Данные}
            \centering
            % {\large На этом слайде уже идут визуальные эффекты}
            % \smallbreak

        \begin{itemize}
            \item Данные из статьи достать не представляется возможным на текущий момент
            \item Использовались пятиминутные данные по биткоину с 06/11/2018 - 03/11/2022
            % \item<3> Этот пункт видно только на третий раз
            % \item<4-> Вот, третий пункт уже отсутствует
        \end{itemize}
    \end{frame}
    \begin{frame}{HAR-модель}
        \begin{itemize}
        \item Семейство HAR-RV моделей (The Heterogeneous Autoregressive model of the Realized Volatility) основывается на гипотезе о гетерогенности рынка
        \item Еще одной особенностью HAR-RV модели является рассмотрение волатильности как ненаблюдаемой
величины, для которой существует наблюдаемая оценка, которую можно оценить при помощи высокочастотных данных. Эта оценка называется реализованной волатильностью
и обозначается RV
        \item Формально является линейной регрессией, где регрессорами выступают недельная и месячная реализованная волатильности
        % \item RV_t = beta_0 + beta_d*RV_t_-_1 + beta_w*RV_t_-_1_:_t_-_5 + beta_m*RV_t_-_1_:_t_-_2_2 + u_t
\end{itemize}
        % Здесь уже пример \pause
        % как делать этот эффект, \pause
        
        % если просто хочешь текст, без itemize
    \end{frame}
    \begin{frame}{Базовый вид, который реализован}
        \begin{itemize}
            \item RV_t = \beta_0 + \beta_d*RV_t_-_1 + \beta_w*RV_t_-_1_:_t_-_5 + \beta_m*RV_t_-_1_:_t_-_2_2 + u_t
        \end{itemize}
        % Можно \alert{выделить} красным важное слово.

        % \begin{block}{Ремарка}
        %     Есть такая вставка
        % \end{block}

        % \begin{alertblock}{Note Bene!}
        %     Есть <<важная>> вставка
        % \end{alertblock}

        % \begin{examples}
        %     Этим блокам тоже можно поменять цвета/шрифт в преамбулу
        % \end{examples}

        % \begin{problem}
            % Здесь можно задачку записать
        % \end{problem}
        % На самом деле есть еще theorem, definition, proof, collorary, example.
    \end{frame}
    \begin{frame}{Ход работы}
        \begin{itemize}
            \item Обработка данных
            \item Расчет дневной, недельной и месячной реализованной волатильности
            \item Логарифмирование
            \item Тест на стационарность
            \item Приведение к стационарности методом первых разностей
            \item Разбиение на train и test
            \item Подбор коэффициентов методом OLS
            \item Значимыми остались являются только дневные и месячные логарифмы реализованной волатильности
        \end{itemize}
        % $$i_t = r^* + \pi^d + \left(1+\frac{\lambda}{\phi(\lambda^2+\alpha)}\right)(\pi^e - \pi^d) + \frac{\lambda}{\phi(\lambda^2+\alpha)} u_t + \frac{1}{\phi}(g_t-y_t^*)$$
        % \begin{columns}
        %     \column{0.05\textwidth}
        %     \column{0.45\textwidth}
        %     Это первая колонка
        %     \begin{itemize}
        %         \item Еще что-то
        %         \item И Ещё
        %     \end{itemize}

        %     \column{0.5\textwidth}
        %     А вот это вторая колонка.\\
        %     Она как-то хитро алайнится
        % \end{columns}
    \end{frame}
    % \begin{frame}{Картинка}
    %     \centering
    %     \includegraphics[scale=0.2]{lion}
    %     % Все о картинках можно узнать тут https://ru.overleaf.com/learn/latex/Inserting_Images
    %     % В том числе как сочетать их с текстом
    % \end{frame}
    \begin{frame}{Промежуточные результаты и метрики}
        \begin{table}[h!]
            \centering
            \begin{tabular}{| l | c | r |}
                \hline
                Метрики & Значения & Пояснения \\
                \hline
                MSE in-sample  & 0.0075 & Ср квадрат ошибки \\
                R-squared in-sample & 0.5865  & Высокий пок-ль \\ [1ex]
                MAE in-sample & 0.0644 & Ср абсолютная ошибка \\
                MSE out-of-sample & 0.0065 & Ср квадрат ошибки\\
                R-squared out-of-sample & 0.4123 & Выоский пок-ль \\
                MAE out-of-sample & 0.0644 & Ср абсолютная ошибка\\
                \hline
            \end{tabular}
        \end{table}
        % Больше про таблицы на https://www.overleaf.com/learn/latex/Tables
    \end{frame}
    \begin{frame}{Дальнейший план}
        \begin{itemize}
            \item Построить другие спецификации: HAR-free, LASSO
            \item Применить методику к высокочастотному VIX (проблема с получением данных)
            \item Сфокусироваться на стратегии имлементации RV для менеджмента позиций. Например, торговать волатильностью с помощью VIX (в процессе обсуждения)
        \end{itemize}
        % Много чего интересного можно прочитать в книгах по использованию Латеха. Например, у Дирака \cite{dirac}. Также есть интересные статьи, например, у Эйнштейна \cite{einstein}. На еще можно посмотреть сайт \cite{knuthwebsite}, и отрывок из книги \cite{knuth-fa}.\\
        % Также можно попробовать самому \href{http://www.overleaf.com}{тут}  или по ссылке: \url{http://www.overleaf.com}
        % % https://www.overleaf.com/learn/latex/Hyperlinks --- ссылки
    \end{frame}

\section{Заключение}
    \begin{frame}{Заключение}
        \begin{itemize}
            \item Подробно рассмотрена одна спецификация
            \item В ближайшее время рассмотреть еще несколько
            \item Реализовать на VIX
        \end{itemize}
        % \centering
        % {\LARGE Надеюсь, этот шаблон будет вам полезен.}\\
        % {\tiny Вы также копировать отдельные элементы (преамбулу, шаблон таблицы, итд) и вставлять их через input.}
    \end{frame}

% \section{Источники}
%     \begin{frame}{Источники}
%         \printbibliography
%     \end{frame}

% \section*{}  
%     \begin{frame}
%         \textcolor{darkblueNES}{\Huge{\centerline{Спасибо за внимание!}}}
%     \end{frame}
\end{document}